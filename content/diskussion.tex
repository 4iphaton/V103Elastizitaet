\newpage
\section{Diskussion}
\label{sec:Diskussion}
Der verwendete Literaturwert ist $E_{lit} = 78 GPa = 78*10^9 Pa$\cite{litval}.
\begin{table}[h]
  \centering
  \label{tab:lit5}
  \begin{tabular}{ c c  }
    \toprule
    $E_{exp} \,\, \text{in} \, [\frac{N}{m^2}= Pa]$
   %&{$E_{lit} \,\, \text{in} \, [Pa]$}
   &{$\text{prozentuale Abweichung}\,\, \text{in} \, \% $} \\

    \midrule


    \bottomrule
  \end{tabular}
  \caption{Abweichung.}
\end{table}


Die bereits zuvor durchgeführte Bestimmung der beiden Metalle lässt darauf
schließen, dass beide Stäbe aus Messing sind. Die Gleichheit des Materials
lässt sich durch die annähernd gleichen Elastizitätsmodule ebenfalls
erkennen.

Die dritte Messung liefert einen unrealistischen und vom einseitig
eingespannten Elastizitätsmodul abweichenden Wert. Zum einen ist dies durch die viel zu geringe Auslenkung
aus der Ruhelage trotz großer Masse zu erklären.

Ein vorliegender systematischer Fehler können die verwendeten Messuhren sein.
Diese beeinflussen sich gegenseitig bei Bewegung und generelle Erschütterungen
der Messoberfläche führen zu einer Verfälschung der Werte. Hinzu kommt, dass
die Messuhren bei gemeinsamen Messungen deutlich unterschiedliche Auslenkungen
anzeigen. Dies beeinflusst vor allem die Messung mit beidseitiger
Einspannung.
Eine weitere Fehleranfälligkeit kann die vorherige Verbiegung des Stabes
sein, welche jedoch durch eine Differenzbildung minimiert werden kann.
