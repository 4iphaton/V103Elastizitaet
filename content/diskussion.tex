\newpage
\section{Diskussion}
\label{sec:Diskussion}
Der verwendete Literaturwert ist $E_{lit} = 78 GPa = 78*10^9 Pa$\cite{litval}.
\begin{table}[h]
  \centering
  \label{tab:lit5}
  \begin{tabular}{ c c c }
    \toprule
    $\text{Messung}$
   &$E_{exp} \,\, \text{in} \, [\frac{N}{m^2}*10^9= GPa]$
   %&{$E_{lit} \,\, \text{in} \, [Pa]$}
   &{$\text{prozentuale Abweichung}\,\, \text{in} \, \si{\percent} $} \\

    \midrule
    1&\num{44.90(15)} & 42.43 \\
    2&\num{42.80(29)} & 45.13  \\
    3&\num{87.59(527)} & 12.30 \\

    \bottomrule
  \end{tabular}
  \caption{Abweichung.}
\end{table}


Die bereits zuvor durchgeführte Bestimmung der Dichte der beiden Metalle lässt darauf
schließen, dass beide Stäbe aus Messing sind. Die Gleichheit des Materials
lässt sich durch die annähernd gleichen Elastizitätsmodule bei einseitiger
Einspannung bestätigen. Der Einfluss vorherigen nicht-reversiblen Biegung
lässt darauf schließen, dass die Biegung der einseitigen Messmethode nicht
komplett elastisch ist, was den gemessenen Elastizitätsmodul beeinflusst.
Die dritte Messung liefert den realistischsten Wert mit einer geringen
Abweichung von \SI{12.3}{\percent}, da diese Messmethode die genauseste ist.
Erklären lässt sich dies durch die geringere Einwirkung des Eigengewichts, da
ein deutlich massereicheres angehängtes Gewicht genutzt wird.
Ein vorliegender systematischer Fehler können die verwendeten Messuhren sein.
Diese beeinflussen sich gegenseitig bei Bewegung und generelle Erschütterungen
der Messoberfläche führen zu einer Verfälschung der Werte. Hinzu kommt, dass
die Messuhren bei gemeinsamen Messungen deutlich unterschiedliche Auslenkungen
anzeigen.
Eine weitere Fehleranfälligkeit kann die vorherige Verbiegung des Stabes
sein, welche jedoch durch eine Differenzbildung minimiert werden kann.
