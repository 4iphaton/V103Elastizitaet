\newpage
\section{Diskussion}
\label{sec:Diskussion}
Der verwendete Literaturwert ist $E_{lit} = 78 GPa = 78*10^9 Pa$\cite{litval}.
\begin{table}[h]
  \centering
  \label{tab:lit5}
  \begin{tabular}{ c c c }
    \toprule
    $\text{Messung}$
   &$E_{exp} \,\, \text{in} \, [\frac{N}{m^2}*10^9= GPa]$
   %&{$E_{lit} \,\, \text{in} \, [Pa]$}
   &{$\text{prozentuale Abweichung}\,\, \text{in} \, \si{\percent} $} \\

    \midrule
    1&\num{88.01(33)} & 12.83\\
    2&\num{84.18(40)} & 7.9  \\
    3&\num{241.17(1299)} & 209.19 \\
    4&\num{190.10(191)} & 143.72\\

    \bottomrule
  \end{tabular}
  \caption{Abweichung.}
\end{table}


Die bereits zuvor durchgeführte Bestimmung der Dichte der beiden Metalle lässt darauf
schließen, dass beide Stäbe aus Messing sind. Die Gleichheit des Materials
lässt sich durch die annähernd gleichen Elastizitätsmodule bei einseitiger
Einspannung bestätigen. Die experimentellen Werte der einseitigen Einspannungen
liegen mit ihrem prozentualen Fehler in einem den Literaturwerten entsprechenden Bereich.
Die große Abweichung des Elastizitätsmoduls bei der beidseitigen Einspannung
kann einerseits durch die gegenseitige Beeinflussung der Messuhren und andererseits
das hohe verwendete Gewicht begründet werden.
%Der Einfluss der vorherigen nicht-reversiblen Biegung
%lässt darauf schließen, dass die Biegung der einseitigen Messmethode nicht
%komplett elastisch ist, was den gemessenen Elastizitätsmodul beeinflusst.
%Die dritte Messung liefert den realistischsten Wert mit einer geringen
%Abweichung von \SI{12.3}{\percent}, da diese Messmethode die genauseste ist.
Ein vorliegender systematischer Fehler können die verwendeten Messuhren sein.
Diese beeinflussen sich nicht nur gegenseitig bei Bewegung und generellen Erschütterungen
der Messoberfläche führen zu einer Verfälschung der Werte. %Hinzu kommt, dass
%die Messuhren bei gemeinsamen Messungen deutlich unterschiedliche Auslenkungen
%anzeigen.
Eine weitere Fehleranfälligkeit kann die vorherige Verbiegung des Stabes
sein, welche jedoch durch eine Differenzbildung minimiert werden kann.
