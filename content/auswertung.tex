\section{Auswertung}
\label{sec:Auswertung}



\subsection{Bestimmung der Metalle über die Dichte}
Die Metalle lassen sich über ihr Volumen V und ihre Masse m bestimmen(Tab.1).
Im Vergleich zu dem Literaturwert von Messing mit $8.41\, \frac{g}{cm^3}$\cite{litval}
zeigt sich, dass beide Stäbe aus Messing bestehen.
\begin{table}[h]
  \centering
  \label{tab:lit}
  \begin{tabular}{ c c c c c c }
    \toprule
    {$\text{Stab}$}
   &{$l \,\, \text{in} \, [m]$}
   &{$d \,\, \text{in} \, [m]$}
   %&{$h \, in \, [m]$}
   &{$V \,\, \text{in} \, [m^3]$}
   &{$m \,\, \text{in} \, [kg]$}
   &{$\rho \,\, \text{in} \, [\frac{g}{cm^3}]$} \\
    \midrule
     {$\text{eckig}$}&60&1&60&502.4 & 8.367 \\
     {$\text{rund}$}&60.05&1&15$\pi$&394.4 & 8.361 \\
    \bottomrule
  \end{tabular}
  \caption{Dichte der Metalle}
\end{table}


\subsection{Eckiger Stab, einseitige Einspannung}

Über die Differenz der Auslenkung  ergibt sich nach
$D(x) = D_0 - D_m$. Die Einspannlänge beträgt $L_1 = 0.493m$ und
das verwendete Gewicht $m_1 = 1.1932 kg$.

\begin{table}[h]
  \centering
  \label{tab:lit2}
  \begin{tabular}{ c c c }
    \toprule
    $x \,\, \text{in} \, [mm]$
   &{$D(x) \,\, \text{in} \, [mm]$}
   &{$(Lx^2- \frac{x^3}{3}) \,\, \text{in} \, [10^{-3}m^3]$} \\

    \midrule
    30  & 0.065& 0.43\\
    50  & 0.135& 1.19\\
    70  & 0.225& 2.30\\
    90  & 0.390& 3.75\\
    110 & 0.530& 5.52\\
    130 & 0.670& 7.60\\
    150 & 0.875& 9.97\\
    170 & 1.085&12.61\\
    190 & 1.320&15.55\\
    210 & 1.560&18.65\\
    230 & 1.810&22.02\\
    250 & 2.105&25.60\\
    270 & 2.390&29.38\\
    290 & 2.675&33.33\\
    310 & 2.990&37.45\\
    330 & 3.340&41.17\\
    350 & 3.735&46.10\\
    370 & 4.038&50.61\\
    390 & 4.480&55.21\\
    410 & 4.740&59.90\\
    430 & 5.145&64.65\\
    450 & 5.470&69.46\\
    470 & 5.860&74.30\\
    480 & 6.070&76.72\\
    \bottomrule
  \end{tabular}
  \caption{Auslenkung eckiger Stab}
\end{table}

\begin{figure}[h]
  \centering
  \includegraphics{build/plot1.pdf}
  \caption{Eckiger Stab, einseitige Einspannung}
  \label{fig:plot1}
\end{figure}



Mit Hilfe von Python wurde eine Regressiongerade berechnet (Abb.3).
Hierbei wird $(Lx^2- \frac{x^3}{3})$ auf der x-Achse gegen $D(x)$ auf der y-Achse
gesetzt, woraus sich die Werte
\begin{align*}
  a &= \SI{8.15(104)e-5}{\per\square\meter} \\
  b &= \SI{7.8205(257)e-2}{\meter} \\
\end{align*}
für die Regressionsgerade ergeben.
\newline
Das vorliegende Flächenträgheitsmoment ist
\begin{equation*}
  I = 1.667* 10^{-9}\, m^4.
\end{equation*}

Hiermit wird der Elastizitätsmodul nach \ref{eqn:einseitig} bestimmt.

\begin{equation*}
  E = \SI{120000(15000)}{\newton\per\square\meter}
\end{equation*}






\subsection{Runder Stab, einseitige Einspannung}
Die Einspannlänge beträgt nun $L_2 = 0.508m$ und es wird ein Gewicht
mit $m_2 = 0.5339 kg $ verwendet.

\begin{table}[h]
  \centering
  \label{tab:lit3}
  \begin{tabular}{ c c c }
    \toprule
    $x \,\, \text{in} \, [mm]$
   &{$D(x) \,\, \text{in} \, [mm]$}
   &{$(Lx^2- \frac{x^3}{3}) \,\, \text{in} \, [10^{-3}m^3]$} \\

    \midrule
    30 & 0.030 & 0.45\\
    50 & 0.110 & 1.23\\
    70 & 0.175 & 2.38\\
    90 & 0.255 & 3.87\\
    110& 0.400 & 5.70\\
    130& 0.540 & 7.85\\
    150& 0.700 &10.31\\
    170& 0.870 &13.04\\
    190& 1.070 &16.05\\
    210& 1.285 &19.32\\
    230& 1.495 &22.82\\
    250& 1.715 &26.54\\
    270& 1.970 &30.47\\
    290& 2.280 &34.59\\
    310& 2.525 &38.89\\
    330& 2.815 &43.34\\
    350& 3.030 &47.94\\
    370& 3.305 &52.66\\
    390& 3.730 &57.49\\
    410& 4.050 &62.42\\
    430& 4.165 &67.43\\
    450& 4.560 &72.50\\
    470& 4.900 &77.61\\
    490& 5.090 &82.75\\

    \bottomrule
  \end{tabular}
  \caption{Auslenkung runder Stab.}
\end{table}

\begin{figure}[h]
  \centering
  \includegraphics{build/plot2.pdf}
  \caption{Runder Stab, einseitige Einspannung.}
  \label{fig:plot2}
\end{figure}

Die erneute Berechnung des Regressionsgeraden über Python
für die vorliegenden Werte ergibt
\begin{align*}
  a &= \SI{5.64(176)e-5}{\per\square\meter} \\
  b &= \SI{6.2320(417)e-2}{\meter} \\
\end{align*}
und das Flächenträgheitsmoment lautet
\begin{equation*}
  I = 9.818* 10^{-10} m^4.
\end{equation*}
\newline
Wodurch sich für den Elastizitätsmodul
\begin{equation*}
  E = \SI{150571.10(4695784)}{\newton\per\square\meter}
\end{equation*}
ergibt.

\subsection{Runder Stab, beidseitige Einspannung}
\begin{figure}
  \centering
  \includegraphics{build/plot3.pdf}
  \caption{Plot3.}
  \label{fig:plot3}
\end{figure}
